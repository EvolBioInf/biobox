\nwfilename{}\nwbegindocs{0}\nwenddocs{}\nwbegindocs{1}\nwdocspar% ===> this file was generated automatically by noweave --- better not edit it
\section*{Introduction}
BLAST calculates local alignments between pairs of sequences and is
used extensively in molecular biology to annotate sequences. The
members of a sequence pair aligned with BLAST are called query and
subject, where the query is searched in the subject. The search
algorithm has three steps, division of the query into short,
overlapping words, $w$, search for the words in the subject, and
extension of matches into alignments. These three steps are
illustrated in Figure~\ref{fig:blast} and we implement them in a
simple BLAST program for DNA sequences, \ty{sblast}.

\begin{figure}
  \begin{center}
    \begin{pspicture}(-0.5,0)(10,3.5)
  \psset{linewidth=1pt,yunit=0.7cm}
%% Subject
%% \rput(-0.5,5.0){$S$}
%% \psline{->}(0.0,5.0)(10.0,5.0)
%% Query
\rput(-0.5,4.5){$Q$}
\psline(0.0,4.5)(5.0,4.5)     
%% Words
\psset{linewidth=0.5pt}
\psline(0.0,4.0)(1.0,4.0)
\psline(0.5,3.8)(1.5,3.8)
\psline(1.0,3.6)(2.0,3.6)
\rput(1.5,3.6){\rnode{s1}{}}
\psline(1.5,3.4)(2.5,3.4)
\psline(2.0,3.2)(3.0,3.2)
\rput(2.5,3.2){\rnode{s2}{}}

\psline(2.5,4.0)(3.5,4.0)
\psline(3.0,3.8)(4.0,3.8)
\psline(3.5,3.6)(4.5,3.6)
\rput(4.0,3.6){\rnode{s3}{}}
\psline(4.0,3.4)(5.0,3.4)

  %% \pscustom{
  %%   \translate(0,-1)
  %% }
  %% Subject
\psset{linewidth=1pt}
\psline{->}(0,1.5)(10,1.5)
\rput(-0.5,1.5){$S$}

\psline[linewidth=0.5pt]{<->}(1.3,1.7)(2.3,1.7)
\rput(1.8,1.7){\rnode{e1}{}}

\psline[linewidth=0.5pt]{<->}(1.9,1.9)(2.9,1.9)
\rput(2.4,1.9){\rnode{e1b}{}}

\psline[linewidth=0.5pt]{<->}(4.2,1.7)(5.2,1.7)
\rput(4.7,1.7){\rnode{e2}{}}

\psline[linewidth=0.5pt]{<->}(7.7,1.7)(8.8,1.7)
\rput(8.2,1.7){\rnode{e3}{}}


  \psset{linecolor=lightgray, linewidth=0.5pt, nodesep=2pt}
  \nccurve[angleA=-120,angleB=90]{->}{s1}{e1}
  \nccurve[angleA=-90,angleB=90]{->}{s2}{e2}
  \nccurve[angleA=-90,angleB=90]{->}{s2}{e1b}
  \nccurve[angleA=-120,angleB=90]{->}{s3}{e3}
\end{pspicture}

  \end{center}
  \caption{Cartoon of the BLAST algorithm.}\label{fig:blast}
\end{figure}

Before we write any code, let's look at the three steps of the
algorithm in a bit more detail starting with the construction of the
word list. Let \ty{GTCGA} be our query and the word length $w=4$, then
the word list is $\{\ty{GTC}, \ty{TCG}, \ty{CGA}\}$. In real
implementations, $w$ is typically at least 11 for DNA sequences. To
emphasize the importance of the word list in the BLAST algorithm, the
user of \ty{sblast} can print it out for inspection.

The query words are looked up in the subject by exact matching using a
keyword tree. This is a tree structure built from the query words. As
illustrated in Chapter~\ref{ch:dkt}, its construction takes some
effort. To persuade the user of \ty{sblast} that this effort is worth
while, we also implement na\"ive matching as an alternative.

Each match of a query word in the subject is extended to the left and
the right until the score of the alignment doesn't grow any
further. Now, a word might be flanked by a mismatch, in which case the
score drops on the first extension, but clearly we shouldn't give up
immediately. So there is a maximum number of extension steps we are
willing to wait for the last maximum score to improve until we give up
and fall back to the position that generated the maximum. We call this
the number of idle extension steps.

This gives us enough understanding of BLAST to get coding.

\section*{Implementation}
Our program outline contains hooks for imports, types, methods,
functions, and the logic of the main function.
\nwenddocs{}\nwbegincode{2}\sublabel{NW0-16JhEl-1}\nwmargintag{{\nwtagstyle{}\subpageref{NW0-16JhEl-1}}}\moddef{sblast.go~{\nwtagstyle{}\subpageref{NW0-16JhEl-1}}}\endmoddef\nwstartdeflinemarkup\nwenddeflinemarkup
package main

import (
          \LA{}Imports, Ch.~\ref{ch:sb}~{\nwtagstyle{}\subpageref{NW0-2Og5jB-1}}\RA{}
)
\LA{}Types, Ch.~\ref{ch:sb}~{\nwtagstyle{}\subpageref{NW0-HzpIk-1}}\RA{}
\LA{}Methods, Ch.~\ref{ch:sb}~{\nwtagstyle{}\subpageref{NW0-HmI1L-1}}\RA{}
\LA{}Functions, Ch.~\ref{ch:sb}~{\nwtagstyle{}\subpageref{NW0-3nuSaQ-1}}\RA{}
func main() \{
          \LA{}Main function, Ch.~\ref{ch:sb}~{\nwtagstyle{}\subpageref{NW0-4WmA2g-1}}\RA{}
\}
\nwnotused{sblast.go}\nwendcode{}\nwbegindocs{3}\nwdocspar
In the main function we set the usage, declare the options, parse the
options, and parse the input files.
\nwenddocs{}\nwbegincode{4}\sublabel{NW0-4WmA2g-1}\nwmargintag{{\nwtagstyle{}\subpageref{NW0-4WmA2g-1}}}\moddef{Main function, Ch.~\ref{ch:sb}~{\nwtagstyle{}\subpageref{NW0-4WmA2g-1}}}\endmoddef\nwstartdeflinemarkup\nwusesondefline{\\{NW0-16JhEl-1}}\nwenddeflinemarkup
\LA{}Set usage, Ch.~\ref{ch:sb}~{\nwtagstyle{}\subpageref{NW0-ys5c2-1}}\RA{}
\LA{}Declare options, Ch.~\ref{ch:sb}~{\nwtagstyle{}\subpageref{NW0-225M11-1}}\RA{}
\LA{}Parse options, Ch.~\ref{ch:sb}~{\nwtagstyle{}\subpageref{NW0-1n3jf6-1}}\RA{}
\LA{}Parse input files, Ch.~\ref{ch:sb}~{\nwtagstyle{}\subpageref{NW0-3CnRTg-1}}\RA{}
\nwused{\\{NW0-16JhEl-1}}\nwendcode{}\nwbegindocs{5}\nwdocspar
The usage consists of the actual usage message, an explanation of the
purpose of \ty{sblast}, and an example command.
\nwenddocs{}\nwbegincode{6}\sublabel{NW0-ys5c2-1}\nwmargintag{{\nwtagstyle{}\subpageref{NW0-ys5c2-1}}}\moddef{Set usage, Ch.~\ref{ch:sb}~{\nwtagstyle{}\subpageref{NW0-ys5c2-1}}}\endmoddef\nwstartdeflinemarkup\nwusesondefline{\\{NW0-4WmA2g-1}}\nwenddeflinemarkup
u := "sblast [-h] [option]... query.fasta [subject.fasta]..."
p := "Carry out a simple version of BLAST."
e := "sblast query.fasta subject.fasta"
clio.Usage(u, p, e)
\nwused{\\{NW0-4WmA2g-1}}\nwendcode{}\nwbegindocs{7}\nwdocspar
We import \ty{clio}.
\nwenddocs{}\nwbegincode{8}\sublabel{NW0-2Og5jB-1}\nwmargintag{{\nwtagstyle{}\subpageref{NW0-2Og5jB-1}}}\moddef{Imports, Ch.~\ref{ch:sb}~{\nwtagstyle{}\subpageref{NW0-2Og5jB-1}}}\endmoddef\nwstartdeflinemarkup\nwusesondefline{\\{NW0-16JhEl-1}}\nwprevnextdefs{\relax}{NW0-2Og5jB-2}\nwenddeflinemarkup
"github.com/evolbioinf/clio"
\nwalsodefined{\\{NW0-2Og5jB-2}\\{NW0-2Og5jB-3}\\{NW0-2Og5jB-4}\\{NW0-2Og5jB-5}\\{NW0-2Og5jB-6}\\{NW0-2Og5jB-7}\\{NW0-2Og5jB-8}\\{NW0-2Og5jB-9}}\nwused{\\{NW0-16JhEl-1}}\nwendcode{}\nwbegindocs{9}\nwdocspar
Apart from help (\ty{-h}), which is already given by the \ty{flag}
package, we provide eight additional options. The algorithm is
specified by match and mismatch scores, the word length, and the
maximum number of idle extension steps. There is a threshold score,
below which an alignment is not printed. The matching method may be
switched to na\"ive and the user can print the word list. These
options and their default values are listed in
Table~\ref{tab:blast}. Wherever I could, I took the defaults from
BLAST.

\begin{table}
  \caption{User options of \ty{sblast} and their defaults.}\label{tab:blast}
  \begin{center}
  \begin{tabular}{clll}
    \hline
    \# & Option & Meaning & Default\\\hline
    1 & \ty{-a} & match & 1\\
    2 & \ty{-i} & mismatch & -3\\
    3 & \ty{-w} & word length & 11\\
    4 & \ty{-s} & idle extension steps & 30\\
    5 & \ty{-t} & threshold score & 50\\
    6 & \ty{-n} & na\"ive matching & false\\
    7 & \ty{-l} & print word list & false\\
    8 & \ty{-v} & print version & false\\\hline
  \end{tabular}
  \end{center}
\end{table}
\nwenddocs{}\nwbegincode{10}\sublabel{NW0-225M11-1}\nwmargintag{{\nwtagstyle{}\subpageref{NW0-225M11-1}}}\moddef{Declare options, Ch.~\ref{ch:sb}~{\nwtagstyle{}\subpageref{NW0-225M11-1}}}\endmoddef\nwstartdeflinemarkup\nwusesondefline{\\{NW0-4WmA2g-1}}\nwenddeflinemarkup
var optA = flag.Float64("a", 1.0, "match")
var optI = flag.Float64("i", -3.0, "mismatch")
var optW = flag.Int("w", 11, "word length")
var optS = flag.Int("s", 30, "maximum number " +
          "of idle extension steps")
var optT = flag.Float64("t", 50.0, "threshold score")
var optN = flag.Bool("n", false, "naive matching")
var optL = flag.Bool("l", false, "print word list")
var optV = flag.Bool("v", false, "print program version " +
          "and other information")
\nwused{\\{NW0-4WmA2g-1}}\nwendcode{}\nwbegindocs{11}\nwdocspar
We import \ty{flag}.
\nwenddocs{}\nwbegincode{12}\sublabel{NW0-2Og5jB-2}\nwmargintag{{\nwtagstyle{}\subpageref{NW0-2Og5jB-2}}}\moddef{Imports, Ch.~\ref{ch:sb}~{\nwtagstyle{}\subpageref{NW0-2Og5jB-1}}}\plusendmoddef\nwstartdeflinemarkup\nwusesondefline{\\{NW0-16JhEl-1}}\nwprevnextdefs{NW0-2Og5jB-1}{NW0-2Og5jB-3}\nwenddeflinemarkup
"flag"
\nwused{\\{NW0-16JhEl-1}}\nwendcode{}\nwbegindocs{13}\nwdocspar
We parse the options and respond to \ty{-v}, as this would terminate
the program. Then we collect the remaining option values.
\nwenddocs{}\nwbegincode{14}\sublabel{NW0-1n3jf6-1}\nwmargintag{{\nwtagstyle{}\subpageref{NW0-1n3jf6-1}}}\moddef{Parse options, Ch.~\ref{ch:sb}~{\nwtagstyle{}\subpageref{NW0-1n3jf6-1}}}\endmoddef\nwstartdeflinemarkup\nwusesondefline{\\{NW0-4WmA2g-1}}\nwenddeflinemarkup
flag.Parse()
if *optV \{
          util.PrintInfo("sblast")
\}
\LA{}Collect option values, Ch.~\ref{ch:sb}~{\nwtagstyle{}\subpageref{NW0-45Dubr-1}}\RA{}
\nwused{\\{NW0-4WmA2g-1}}\nwendcode{}\nwbegindocs{15}\nwdocspar
We import \ty{util}.
\nwenddocs{}\nwbegincode{16}\sublabel{NW0-2Og5jB-3}\nwmargintag{{\nwtagstyle{}\subpageref{NW0-2Og5jB-3}}}\moddef{Imports, Ch.~\ref{ch:sb}~{\nwtagstyle{}\subpageref{NW0-2Og5jB-1}}}\plusendmoddef\nwstartdeflinemarkup\nwusesondefline{\\{NW0-16JhEl-1}}\nwprevnextdefs{NW0-2Og5jB-2}{NW0-2Og5jB-4}\nwenddeflinemarkup
"github.com/evolbioinf/biobox/util"
\nwused{\\{NW0-16JhEl-1}}\nwendcode{}\nwbegindocs{17}\nwdocspar
There are seven options we later pass to the BLAST algorithm. To
make this easy, we collect them in the variable \ty{opts}.
\nwenddocs{}\nwbegincode{18}\sublabel{NW0-45Dubr-1}\nwmargintag{{\nwtagstyle{}\subpageref{NW0-45Dubr-1}}}\moddef{Collect option values, Ch.~\ref{ch:sb}~{\nwtagstyle{}\subpageref{NW0-45Dubr-1}}}\endmoddef\nwstartdeflinemarkup\nwusesondefline{\\{NW0-1n3jf6-1}}\nwenddeflinemarkup
opts := new(Opts)
opts.a = *optA
opts.i = *optI
opts.w = *optW
opts.s = *optS
opts.t = *optT
opts.n = *optN
opts.l = *optL
\nwused{\\{NW0-1n3jf6-1}}\nwendcode{}\nwbegindocs{19}\nwdocspar
We declare the type \ty{Opts}.
\nwenddocs{}\nwbegincode{20}\sublabel{NW0-HzpIk-1}\nwmargintag{{\nwtagstyle{}\subpageref{NW0-HzpIk-1}}}\moddef{Types, Ch.~\ref{ch:sb}~{\nwtagstyle{}\subpageref{NW0-HzpIk-1}}}\endmoddef\nwstartdeflinemarkup\nwusesondefline{\\{NW0-16JhEl-1}}\nwprevnextdefs{\relax}{NW0-HzpIk-2}\nwenddeflinemarkup
type Opts struct \{
          a, i, t float64
          w, s int
          n, l bool
\}
\nwalsodefined{\\{NW0-HzpIk-2}\\{NW0-HzpIk-3}\\{NW0-HzpIk-4}\\{NW0-HzpIk-5}}\nwused{\\{NW0-16JhEl-1}}\nwendcode{}\nwbegindocs{21}\nwdocspar
The remaining tokens on the command line are taken as the names of
input files. The first of these contains the query sequences, any
subsequent file the subject sequences. If there is no query file, we
bail with a friendly message. If there is, we call \ty{ParseFiles},
which has as first parameter the names of the subject files, and
second parameter the function \ty{scan}. This function is applied to
each subject file and takes as arguments the options and the query
file. It also takes as argument a tab writer to align the columns of
the output. This is initialized with the column headers and flushed
after the run is finished.
\nwenddocs{}\nwbegincode{22}\sublabel{NW0-3CnRTg-1}\nwmargintag{{\nwtagstyle{}\subpageref{NW0-3CnRTg-1}}}\moddef{Parse input files, Ch.~\ref{ch:sb}~{\nwtagstyle{}\subpageref{NW0-3CnRTg-1}}}\endmoddef\nwstartdeflinemarkup\nwusesondefline{\\{NW0-4WmA2g-1}}\nwenddeflinemarkup
files := flag.Args()
if len(files) == 0 \{
          log.Fatal("please provide a query")
\}
out := tabwriter.NewWriter(os.Stdout, 2, 1, 2, ' ' ,0)
if !opts.l \{
          fmt.Fprintf(out, "#qa\\tsa\\tqs\\tqe\\tsa\\tse\\tscore\\n")
\} else \{
          fmt.Fprintf(out, "#qa\\tn\\tword\\n")
\}
clio.ParseFiles(files[1:], scan, opts, files[0], out)
out.Flush()
\nwused{\\{NW0-4WmA2g-1}}\nwendcode{}\nwbegindocs{23}\nwdocspar
We import \ty{tabwriter} and \ty{fmt}.
\nwenddocs{}\nwbegincode{24}\sublabel{NW0-2Og5jB-4}\nwmargintag{{\nwtagstyle{}\subpageref{NW0-2Og5jB-4}}}\moddef{Imports, Ch.~\ref{ch:sb}~{\nwtagstyle{}\subpageref{NW0-2Og5jB-1}}}\plusendmoddef\nwstartdeflinemarkup\nwusesondefline{\\{NW0-16JhEl-1}}\nwprevnextdefs{NW0-2Og5jB-3}{NW0-2Og5jB-5}\nwenddeflinemarkup
"text/tabwriter"
"fmt"
\nwused{\\{NW0-16JhEl-1}}\nwendcode{}\nwbegindocs{25}\nwdocspar
Inside \ty{scan}, we retrieve the arguments, iterate across the
subject sequences, and for each one iterate across the queries.
\nwenddocs{}\nwbegincode{26}\sublabel{NW0-3nuSaQ-1}\nwmargintag{{\nwtagstyle{}\subpageref{NW0-3nuSaQ-1}}}\moddef{Functions, Ch.~\ref{ch:sb}~{\nwtagstyle{}\subpageref{NW0-3nuSaQ-1}}}\endmoddef\nwstartdeflinemarkup\nwusesondefline{\\{NW0-16JhEl-1}}\nwprevnextdefs{\relax}{NW0-3nuSaQ-2}\nwenddeflinemarkup
func scan(r io.Reader, args ...interface\{\}) \{
          \LA{}Retrieve arguments, Ch.~\ref{ch:sb}~{\nwtagstyle{}\subpageref{NW0-3worYP-1}}\RA{}
          sScanner := fasta.NewScanner(r)
          for sScanner.ScanSequence() \{
                  subject := sScanner.Sequence()
                  \LA{}Iterate across queries, Ch.~\ref{ch:sb}~{\nwtagstyle{}\subpageref{NW0-2otCSW-1}}\RA{}
          \}
\}
\nwalsodefined{\\{NW0-3nuSaQ-2}\\{NW0-3nuSaQ-3}}\nwused{\\{NW0-16JhEl-1}}\nwendcode{}\nwbegindocs{27}\nwdocspar
We import \ty{io} and \ty{fasta}.
\nwenddocs{}\nwbegincode{28}\sublabel{NW0-2Og5jB-5}\nwmargintag{{\nwtagstyle{}\subpageref{NW0-2Og5jB-5}}}\moddef{Imports, Ch.~\ref{ch:sb}~{\nwtagstyle{}\subpageref{NW0-2Og5jB-1}}}\plusendmoddef\nwstartdeflinemarkup\nwusesondefline{\\{NW0-16JhEl-1}}\nwprevnextdefs{NW0-2Og5jB-4}{NW0-2Og5jB-6}\nwenddeflinemarkup
"io"
"github.com/evolbioinf/fasta"
\nwused{\\{NW0-16JhEl-1}}\nwendcode{}\nwbegindocs{29}\nwdocspar
The options and the queries are retrieved by type assertion.
\nwenddocs{}\nwbegincode{30}\sublabel{NW0-3worYP-1}\nwmargintag{{\nwtagstyle{}\subpageref{NW0-3worYP-1}}}\moddef{Retrieve arguments, Ch.~\ref{ch:sb}~{\nwtagstyle{}\subpageref{NW0-3worYP-1}}}\endmoddef\nwstartdeflinemarkup\nwusesondefline{\\{NW0-3nuSaQ-1}}\nwenddeflinemarkup
opts := args[0].(*Opts)
qName := args[1].(string)
out := args[2].(*tabwriter.Writer)
\nwused{\\{NW0-3nuSaQ-1}}\nwendcode{}\nwbegindocs{31}\nwdocspar
We open the query file and analyze each sequence it contains.
\nwenddocs{}\nwbegincode{32}\sublabel{NW0-2otCSW-1}\nwmargintag{{\nwtagstyle{}\subpageref{NW0-2otCSW-1}}}\moddef{Iterate across queries, Ch.~\ref{ch:sb}~{\nwtagstyle{}\subpageref{NW0-2otCSW-1}}}\endmoddef\nwstartdeflinemarkup\nwusesondefline{\\{NW0-3nuSaQ-1}}\nwenddeflinemarkup
qFile, err := os.Open(qName)
if err != nil \{
          log.Fatalf("couldn't open %s\\n", qName)
\}
defer qFile.Close()
qScanner := fasta.NewScanner(qFile)
for qScanner.ScanSequence() \{
          query := qScanner.Sequence()
          \LA{}Analyze query, Ch.~\ref{ch:sb}~{\nwtagstyle{}\subpageref{NW0-3uTedU-1}}\RA{}
\}
\nwused{\\{NW0-3nuSaQ-1}}\nwendcode{}\nwbegindocs{33}\nwdocspar
We import \ty{os} and \ty{log}.
\nwenddocs{}\nwbegincode{34}\sublabel{NW0-2Og5jB-6}\nwmargintag{{\nwtagstyle{}\subpageref{NW0-2Og5jB-6}}}\moddef{Imports, Ch.~\ref{ch:sb}~{\nwtagstyle{}\subpageref{NW0-2Og5jB-1}}}\plusendmoddef\nwstartdeflinemarkup\nwusesondefline{\\{NW0-16JhEl-1}}\nwprevnextdefs{NW0-2Og5jB-5}{NW0-2Og5jB-7}\nwenddeflinemarkup
"os"
"log"
\nwused{\\{NW0-16JhEl-1}}\nwendcode{}\nwbegindocs{35}\nwdocspar
A query either gets its word list printed or is aligned to the
subject.
\nwenddocs{}\nwbegincode{36}\sublabel{NW0-3uTedU-1}\nwmargintag{{\nwtagstyle{}\subpageref{NW0-3uTedU-1}}}\moddef{Analyze query, Ch.~\ref{ch:sb}~{\nwtagstyle{}\subpageref{NW0-3uTedU-1}}}\endmoddef\nwstartdeflinemarkup\nwusesondefline{\\{NW0-2otCSW-1}}\nwenddeflinemarkup
if opts.l \{
          \LA{}Print word list, Ch.~\ref{ch:sb}~{\nwtagstyle{}\subpageref{NW0-4ax9cN-1}}\RA{}
\} else \{
          \LA{}Align query, Ch.~\ref{ch:sb}~{\nwtagstyle{}\subpageref{NW0-1R6yr3-1}}\RA{}
\}
\nwused{\\{NW0-2otCSW-1}}\nwendcode{}\nwbegindocs{37}\nwdocspar
A word list is started by the header of the sequence. The list itself
consists of numbered words, one per line. We only write the words on
the forward strand. Since we might write the word lists for more than
one query, we extract the query accession as the first token on the
command line.
\nwenddocs{}\nwbegincode{38}\sublabel{NW0-4ax9cN-1}\nwmargintag{{\nwtagstyle{}\subpageref{NW0-4ax9cN-1}}}\moddef{Print word list, Ch.~\ref{ch:sb}~{\nwtagstyle{}\subpageref{NW0-4ax9cN-1}}}\endmoddef\nwstartdeflinemarkup\nwusesondefline{\\{NW0-3uTedU-1}}\nwenddeflinemarkup
words := getWords(query, opts.w)
qa := strings.Fields(query.Header())[0]
for i, word := range words \{
          fmt.Fprintf(out, "%s\\t%d\\t%s\\n", qa, i+1, word)
\}
\nwused{\\{NW0-3uTedU-1}}\nwendcode{}\nwbegindocs{39}\nwdocspar
We import \ty{strings}.
\nwenddocs{}\nwbegincode{40}\sublabel{NW0-2Og5jB-7}\nwmargintag{{\nwtagstyle{}\subpageref{NW0-2Og5jB-7}}}\moddef{Imports, Ch.~\ref{ch:sb}~{\nwtagstyle{}\subpageref{NW0-2Og5jB-1}}}\plusendmoddef\nwstartdeflinemarkup\nwusesondefline{\\{NW0-16JhEl-1}}\nwprevnextdefs{NW0-2Og5jB-6}{NW0-2Og5jB-8}\nwenddeflinemarkup
"strings"
\nwused{\\{NW0-16JhEl-1}}\nwendcode{}\nwbegindocs{41}\nwdocspar
The function \ty{getWords} takes as argument a sequence and a word
length and returns all words of that length.
\nwenddocs{}\nwbegincode{42}\sublabel{NW0-3nuSaQ-2}\nwmargintag{{\nwtagstyle{}\subpageref{NW0-3nuSaQ-2}}}\moddef{Functions, Ch.~\ref{ch:sb}~{\nwtagstyle{}\subpageref{NW0-3nuSaQ-1}}}\plusendmoddef\nwstartdeflinemarkup\nwusesondefline{\\{NW0-16JhEl-1}}\nwprevnextdefs{NW0-3nuSaQ-1}{NW0-3nuSaQ-3}\nwenddeflinemarkup
func getWords(seq *fasta.Sequence, w int) []string \{
          var words []string
          d := seq.Data()
          l := len(d)
          for i := 0; i <= l - w; i++ \{
                  word := string(d[i:i+w])
                  words = append(words, word)
          \}
          return words
\}
\nwused{\\{NW0-16JhEl-1}}\nwendcode{}\nwbegindocs{43}\nwdocspar
We align the query first along its forward strand, then along its
reverse strand. We print the resulting alignments.
\nwenddocs{}\nwbegincode{44}\sublabel{NW0-1R6yr3-1}\nwmargintag{{\nwtagstyle{}\subpageref{NW0-1R6yr3-1}}}\moddef{Align query, Ch.~\ref{ch:sb}~{\nwtagstyle{}\subpageref{NW0-1R6yr3-1}}}\endmoddef\nwstartdeflinemarkup\nwusesondefline{\\{NW0-3uTedU-1}}\nwenddeflinemarkup
forward := true
alignments := align(query, subject, opts, forward)
query.ReverseComplement()
forward = false
a := align(query, subject, opts, forward)
alignments = append(alignments, a...)
\LA{}Print alignments, Ch.~\ref{ch:sb}~{\nwtagstyle{}\subpageref{NW0-4aXeEl-1}}\RA{}
\nwused{\\{NW0-3uTedU-1}}\nwendcode{}\nwbegindocs{45}\nwdocspar
Inside the function \ty{align}, we calculate the alignments and return
them.
\nwenddocs{}\nwbegincode{46}\sublabel{NW0-3nuSaQ-3}\nwmargintag{{\nwtagstyle{}\subpageref{NW0-3nuSaQ-3}}}\moddef{Functions, Ch.~\ref{ch:sb}~{\nwtagstyle{}\subpageref{NW0-3nuSaQ-1}}}\plusendmoddef\nwstartdeflinemarkup\nwusesondefline{\\{NW0-16JhEl-1}}\nwprevnextdefs{NW0-3nuSaQ-2}{\relax}\nwenddeflinemarkup
func align(query, subject *fasta.Sequence,
          opts *Opts, forward bool) []Alignment \{
          var alignments []Alignment
          \LA{}Calculate alignments, Ch.~\ref{ch:sb}~{\nwtagstyle{}\subpageref{NW0-1DYjD7-1}}\RA{}
          return alignments
\}
\nwused{\\{NW0-16JhEl-1}}\nwendcode{}\nwbegindocs{47}\nwdocspar
An alignment consists of query start and end, subject start and end,
a score, and a strand.
\nwenddocs{}\nwbegincode{48}\sublabel{NW0-HzpIk-2}\nwmargintag{{\nwtagstyle{}\subpageref{NW0-HzpIk-2}}}\moddef{Types, Ch.~\ref{ch:sb}~{\nwtagstyle{}\subpageref{NW0-HzpIk-1}}}\plusendmoddef\nwstartdeflinemarkup\nwusesondefline{\\{NW0-16JhEl-1}}\nwprevnextdefs{NW0-HzpIk-1}{NW0-HzpIk-3}\nwenddeflinemarkup
type Alignment struct \{
          qs, qe, ss, se int
          score float64
          forward bool
\}
\nwused{\\{NW0-16JhEl-1}}\nwendcode{}\nwbegindocs{49}\nwdocspar
As shown in Figure~\ref{fig:blast}, we initialize alignments through
exact matching and then extend the matches to the left and to the
right. Then we filter the alignments and sort them by score.
\nwenddocs{}\nwbegincode{50}\sublabel{NW0-1DYjD7-1}\nwmargintag{{\nwtagstyle{}\subpageref{NW0-1DYjD7-1}}}\moddef{Calculate alignments, Ch.~\ref{ch:sb}~{\nwtagstyle{}\subpageref{NW0-1DYjD7-1}}}\endmoddef\nwstartdeflinemarkup\nwusesondefline{\\{NW0-3nuSaQ-3}}\nwenddeflinemarkup
\LA{}Exact matching, Ch.~\ref{ch:sb}~{\nwtagstyle{}\subpageref{NW0-1uN9bQ-1}}\RA{}
\LA{}Extend alignments, Ch.~\ref{ch:sb}~{\nwtagstyle{}\subpageref{NW0-2p7uJM-1}}\RA{}
\LA{}Filter alignments, Ch.~\ref{ch:sb}~{\nwtagstyle{}\subpageref{NW0-rI1tg-1}}\RA{}
\LA{}Sort alignments by score, Ch.~\ref{ch:sb}~{\nwtagstyle{}\subpageref{NW0-2cS5wy-1}}\RA{}
\nwused{\\{NW0-3nuSaQ-3}}\nwendcode{}\nwbegindocs{51}\nwdocspar
As shown in Figure~\ref{fig:blast}, in the exact matching phase of the
algorithm query words are located in the subject. We store these
matches as mini alignments, which we either find by na\"ive matching
or by matching with a keyword tree.
\nwenddocs{}\nwbegincode{52}\sublabel{NW0-1uN9bQ-1}\nwmargintag{{\nwtagstyle{}\subpageref{NW0-1uN9bQ-1}}}\moddef{Exact matching, Ch.~\ref{ch:sb}~{\nwtagstyle{}\subpageref{NW0-1uN9bQ-1}}}\endmoddef\nwstartdeflinemarkup\nwusesondefline{\\{NW0-1DYjD7-1}}\nwenddeflinemarkup
if opts.n \{
          \LA{}Na\"ive exact matching, Ch.~\ref{ch:sb}~{\nwtagstyle{}\subpageref{NW0-4CQMjJ-1}}\RA{}
\} else \{
          \LA{}Exact match with keyword tree, Ch.~\ref{ch:sb}~{\nwtagstyle{}\subpageref{NW0-2zPa6i-1}}\RA{}
\}
\nwused{\\{NW0-1DYjD7-1}}\nwendcode{}\nwbegindocs{53}\nwdocspar
In na\"ive exact matching, we iterate over the query to generate the
patterns and then look for them in the subject.
\nwenddocs{}\nwbegincode{54}\sublabel{NW0-4CQMjJ-1}\nwmargintag{{\nwtagstyle{}\subpageref{NW0-4CQMjJ-1}}}\moddef{Na\"ive exact matching, Ch.~\ref{ch:sb}~{\nwtagstyle{}\subpageref{NW0-4CQMjJ-1}}}\endmoddef\nwstartdeflinemarkup\nwusesondefline{\\{NW0-1uN9bQ-1}}\nwenddeflinemarkup
q := query.Data()
m := len(q)
s := subject.Data()
n := len(s)
w := opts.w
for i := 0; i < m - w; i++ \{
          p := q[i:i+w]
          for j := 0; j < n - w; j++ \{
                  \LA{}Look for pattern, Ch.~\ref{ch:sb}~{\nwtagstyle{}\subpageref{NW0-1aqFvq-1}}\RA{}
          \}
\}
\nwused{\\{NW0-1uN9bQ-1}}\nwendcode{}\nwbegindocs{55}\nwdocspar
We break off the search for a pattern at the first mismatch we
encounter.
\nwenddocs{}\nwbegincode{56}\sublabel{NW0-1aqFvq-1}\nwmargintag{{\nwtagstyle{}\subpageref{NW0-1aqFvq-1}}}\moddef{Look for pattern, Ch.~\ref{ch:sb}~{\nwtagstyle{}\subpageref{NW0-1aqFvq-1}}}\endmoddef\nwstartdeflinemarkup\nwusesondefline{\\{NW0-4CQMjJ-1}}\nwenddeflinemarkup
var k int
for k = 0; k < w; k++ \{
          if s[j+k] != p[k] \{
                  break
          \}
\}
if k == opts.w \{
          a := Alignment\{qs: i, qe: i+w-1, ss: j, se: j+w-1,
                  score: float64(w) *opts.a, forward: forward\}
          alignments = append(alignments, a)
\}
\nwused{\\{NW0-4CQMjJ-1}}\nwendcode{}\nwbegindocs{57}\nwdocspar
With a keyword tree, we look for all patterns at the same time. So we
construct the patterns and their keyword tree, search for matches in
the subject, and store the matches as alignments.
\nwenddocs{}\nwbegincode{58}\sublabel{NW0-2zPa6i-1}\nwmargintag{{\nwtagstyle{}\subpageref{NW0-2zPa6i-1}}}\moddef{Exact match with keyword tree, Ch.~\ref{ch:sb}~{\nwtagstyle{}\subpageref{NW0-2zPa6i-1}}}\endmoddef\nwstartdeflinemarkup\nwusesondefline{\\{NW0-1uN9bQ-1}}\nwenddeflinemarkup
\LA{}Construct patterns, Ch.~\ref{ch:sb}~{\nwtagstyle{}\subpageref{NW0-UZ3nD-1}}\RA{}
\LA{}Construct keyword tree, Ch.~\ref{ch:sb}~{\nwtagstyle{}\subpageref{NW0-4Wz7Gt-1}}\RA{}
\LA{}Search with keyword tree, Ch.~\ref{ch:sb}~{\nwtagstyle{}\subpageref{NW0-1YDNRQ-1}}\RA{}
\LA{}Convert matches to alignments, Ch.~\ref{ch:sb}~{\nwtagstyle{}\subpageref{NW0-2JbbBG-1}}\RA{}
\nwused{\\{NW0-1uN9bQ-1}}\nwendcode{}\nwbegindocs{59}\nwdocspar
We store the patterns as a string slice.
\nwenddocs{}\nwbegincode{60}\sublabel{NW0-UZ3nD-1}\nwmargintag{{\nwtagstyle{}\subpageref{NW0-UZ3nD-1}}}\moddef{Construct patterns, Ch.~\ref{ch:sb}~{\nwtagstyle{}\subpageref{NW0-UZ3nD-1}}}\endmoddef\nwstartdeflinemarkup\nwusesondefline{\\{NW0-2zPa6i-1}}\nwenddeflinemarkup
var patterns []string
q := query.Data()
m := len(q)
w := opts.w
for i := 0; i <= m-w; i++ \{
          p := string(q[i:i+w])
          patterns = append(patterns, p)
\}
\nwused{\\{NW0-2zPa6i-1}}\nwendcode{}\nwbegindocs{61}\nwdocspar
The keyword tree is constructed by a function call.
\nwenddocs{}\nwbegincode{62}\sublabel{NW0-4Wz7Gt-1}\nwmargintag{{\nwtagstyle{}\subpageref{NW0-4Wz7Gt-1}}}\moddef{Construct keyword tree, Ch.~\ref{ch:sb}~{\nwtagstyle{}\subpageref{NW0-4Wz7Gt-1}}}\endmoddef\nwstartdeflinemarkup\nwusesondefline{\\{NW0-2zPa6i-1}}\nwenddeflinemarkup
tree := kt.NewKeywordTree(patterns)
\nwused{\\{NW0-2zPa6i-1}}\nwendcode{}\nwbegindocs{63}\nwdocspar
We import \ty{kt}.
\nwenddocs{}\nwbegincode{64}\sublabel{NW0-2Og5jB-8}\nwmargintag{{\nwtagstyle{}\subpageref{NW0-2Og5jB-8}}}\moddef{Imports, Ch.~\ref{ch:sb}~{\nwtagstyle{}\subpageref{NW0-2Og5jB-1}}}\plusendmoddef\nwstartdeflinemarkup\nwusesondefline{\\{NW0-16JhEl-1}}\nwprevnextdefs{NW0-2Og5jB-7}{NW0-2Og5jB-9}\nwenddeflinemarkup
"github.com/evolbioinf/kt"
\nwused{\\{NW0-16JhEl-1}}\nwendcode{}\nwbegindocs{65}\nwdocspar
The search with the keyword tree is also a single function call.
\nwenddocs{}\nwbegincode{66}\sublabel{NW0-1YDNRQ-1}\nwmargintag{{\nwtagstyle{}\subpageref{NW0-1YDNRQ-1}}}\moddef{Search with keyword tree, Ch.~\ref{ch:sb}~{\nwtagstyle{}\subpageref{NW0-1YDNRQ-1}}}\endmoddef\nwstartdeflinemarkup\nwusesondefline{\\{NW0-2zPa6i-1}}\nwenddeflinemarkup
matches := tree.Search(subject.Data(), patterns)
\nwused{\\{NW0-2zPa6i-1}}\nwendcode{}\nwbegindocs{67}\nwdocspar
We iterate over the matches and convert them to our proto alignments.
\nwenddocs{}\nwbegincode{68}\sublabel{NW0-2JbbBG-1}\nwmargintag{{\nwtagstyle{}\subpageref{NW0-2JbbBG-1}}}\moddef{Convert matches to alignments, Ch.~\ref{ch:sb}~{\nwtagstyle{}\subpageref{NW0-2JbbBG-1}}}\endmoddef\nwstartdeflinemarkup\nwusesondefline{\\{NW0-2zPa6i-1}}\nwenddeflinemarkup
for _, m := range matches \{
          qs := m.Pattern
          ss := m.Position
          qe := qs + w - 1
          se := ss + w - 1
          sc := float64(w) * opts.a
          a := Alignment\{qs: qs, ss: ss, qe: qe,
                  se: se, score: sc, forward: forward\}
          alignments = append(alignments, a)
\}
\nwused{\\{NW0-2zPa6i-1}}\nwendcode{}\nwbegindocs{69}\nwdocspar
We extend each alignment seed by walking to the left and to the right.
\nwenddocs{}\nwbegincode{70}\sublabel{NW0-2p7uJM-1}\nwmargintag{{\nwtagstyle{}\subpageref{NW0-2p7uJM-1}}}\moddef{Extend alignments, Ch.~\ref{ch:sb}~{\nwtagstyle{}\subpageref{NW0-2p7uJM-1}}}\endmoddef\nwstartdeflinemarkup\nwusesondefline{\\{NW0-1DYjD7-1}}\nwenddeflinemarkup
q := query.Data()
m := len(q)
s := subject.Data()
n := len(s)
for i, _ := range alignments \{
          \LA{}Walk left, Ch.~\ref{ch:sb}~{\nwtagstyle{}\subpageref{NW0-35cUih-1}}\RA{}
          \LA{}Walk right, Ch.~\ref{ch:sb}~{\nwtagstyle{}\subpageref{NW0-3FnixI-1}}\RA{}
\}
\nwused{\\{NW0-1DYjD7-1}}\nwendcode{}\nwbegindocs{71}\nwdocspar
We walk left until we run out of query or subject, or until we run out
of idle steps. In each step we compare the current pair of residues
and ask whether we should adjust the alignment start.
\nwenddocs{}\nwbegincode{72}\sublabel{NW0-35cUih-1}\nwmargintag{{\nwtagstyle{}\subpageref{NW0-35cUih-1}}}\moddef{Walk left, Ch.~\ref{ch:sb}~{\nwtagstyle{}\subpageref{NW0-35cUih-1}}}\endmoddef\nwstartdeflinemarkup\nwusesondefline{\\{NW0-2p7uJM-1}}\nwenddeflinemarkup
cq := alignments[i].qs - 1
cs := alignments[i].ss - 1
score := alignments[i].score
is := 0
for cq >= 0 && cs >= 0 && is <= opts.s \{
          \LA{}Compare current pair of residues, Ch.~\ref{ch:sb}~{\nwtagstyle{}\subpageref{NW0-4WxDgc-1}}\RA{}
          \LA{}Adjust alignment start? Ch.~\ref{ch:sb}~{\nwtagstyle{}\subpageref{NW0-vHJSe-1}}\RA{}
          cq--
          cs--
\}
\nwused{\\{NW0-2p7uJM-1}}\nwendcode{}\nwbegindocs{73}\nwdocspar
If a pair of residues is identical, we add the match score to the
current score, otherwise we add the mismatch score.
\nwenddocs{}\nwbegincode{74}\sublabel{NW0-4WxDgc-1}\nwmargintag{{\nwtagstyle{}\subpageref{NW0-4WxDgc-1}}}\moddef{Compare current pair of residues, Ch.~\ref{ch:sb}~{\nwtagstyle{}\subpageref{NW0-4WxDgc-1}}}\endmoddef\nwstartdeflinemarkup\nwusesondefline{\\{NW0-35cUih-1}\\{NW0-3FnixI-1}}\nwenddeflinemarkup
if q[cq] == s[cs] \{
          score += opts.a
\} else \{
          score += opts.i
\}
\nwused{\\{NW0-35cUih-1}\\{NW0-3FnixI-1}}\nwendcode{}\nwbegindocs{75}\nwdocspar
If the alignment score has grown, we shift the alignment start to the
left, set the new maximum score, and reset the number of idle steps to
zero. Otherwise, we've just carried out an idle step.
\nwenddocs{}\nwbegincode{76}\sublabel{NW0-vHJSe-1}\nwmargintag{{\nwtagstyle{}\subpageref{NW0-vHJSe-1}}}\moddef{Adjust alignment start? Ch.~\ref{ch:sb}~{\nwtagstyle{}\subpageref{NW0-vHJSe-1}}}\endmoddef\nwstartdeflinemarkup\nwusesondefline{\\{NW0-35cUih-1}}\nwenddeflinemarkup
if score > alignments[i].score \{
          alignments[i].score = score
          alignments[i].qs = cq
          alignments[i].ss = cs
          is = 0
\} else \{
          is++
\}
\nwused{\\{NW0-35cUih-1}}\nwendcode{}\nwbegindocs{77}\nwdocspar
Walking to the right is similar as walking to the left, except that
now we ask whether we should adjust the alignment end.
\nwenddocs{}\nwbegincode{78}\sublabel{NW0-3FnixI-1}\nwmargintag{{\nwtagstyle{}\subpageref{NW0-3FnixI-1}}}\moddef{Walk right, Ch.~\ref{ch:sb}~{\nwtagstyle{}\subpageref{NW0-3FnixI-1}}}\endmoddef\nwstartdeflinemarkup\nwusesondefline{\\{NW0-2p7uJM-1}}\nwenddeflinemarkup
cq = alignments[i].qe + 1
cs = alignments[i].se + 1
score = alignments[i].score
is = 0
for cq < m && cs < n && is <= opts.s \{
          \LA{}Compare current pair of residues, Ch.~\ref{ch:sb}~{\nwtagstyle{}\subpageref{NW0-4WxDgc-1}}\RA{}
          \LA{}Adjust alignment end? Ch.~\ref{ch:sb}~{\nwtagstyle{}\subpageref{NW0-3W96lC-1}}\RA{}
          cq++
          cs++
\}
\nwused{\\{NW0-2p7uJM-1}}\nwendcode{}\nwbegindocs{79}\nwdocspar
If the score has improved, we extend the alignment to the right, set
the new score, and reset the number of idle steps to zero. Otherwise,
we increment the idle steps.
\nwenddocs{}\nwbegincode{80}\sublabel{NW0-3W96lC-1}\nwmargintag{{\nwtagstyle{}\subpageref{NW0-3W96lC-1}}}\moddef{Adjust alignment end? Ch.~\ref{ch:sb}~{\nwtagstyle{}\subpageref{NW0-3W96lC-1}}}\endmoddef\nwstartdeflinemarkup\nwusesondefline{\\{NW0-3FnixI-1}}\nwenddeflinemarkup
if score > alignments[i].score \{
          alignments[i].score = score
          alignments[i].qe = cq
          alignments[i].se = cs
          is = 0
\} else \{
          is++
\}
\nwused{\\{NW0-3FnixI-1}}\nwendcode{}\nwbegindocs{81}\nwdocspar
We filter the alignments by removing those with low scores. In
addition, words that land in the same homologous region on the
subject may generate alignments that are contained in each
other. We remove these redundant alignments.
\nwenddocs{}\nwbegincode{82}\sublabel{NW0-rI1tg-1}\nwmargintag{{\nwtagstyle{}\subpageref{NW0-rI1tg-1}}}\moddef{Filter alignments, Ch.~\ref{ch:sb}~{\nwtagstyle{}\subpageref{NW0-rI1tg-1}}}\endmoddef\nwstartdeflinemarkup\nwusesondefline{\\{NW0-1DYjD7-1}}\nwenddeflinemarkup
\LA{}Remove low-scoring alignments, Ch.~\ref{ch:sb}~{\nwtagstyle{}\subpageref{NW0-1lD71o-1}}\RA{}
\LA{}Remove redundant alignments, Ch.~\ref{ch:sb}~{\nwtagstyle{}\subpageref{NW0-uGjcy-1}}\RA{}
\nwused{\\{NW0-1DYjD7-1}}\nwendcode{}\nwbegindocs{83}\nwdocspar
We keep only alignments with a score greater or equal to the threshold
score.
\nwenddocs{}\nwbegincode{84}\sublabel{NW0-1lD71o-1}\nwmargintag{{\nwtagstyle{}\subpageref{NW0-1lD71o-1}}}\moddef{Remove low-scoring alignments, Ch.~\ref{ch:sb}~{\nwtagstyle{}\subpageref{NW0-1lD71o-1}}}\endmoddef\nwstartdeflinemarkup\nwusesondefline{\\{NW0-rI1tg-1}}\nwenddeflinemarkup
i := 0
max := -1.0
for _, al := range alignments \{
          if al.score >= opts.t \{
                  if max < al.score \{ max = al.score \}
                  alignments[i] = al
                  i++
          \}
\}
alignments = alignments[:i]
\nwused{\\{NW0-rI1tg-1}}\nwendcode{}\nwbegindocs{85}\nwdocspar
Redundant alignments tend to either share a start position or an end
position. So we sort by start position as primary key and reduce runs
of identical start positions to the first element. This means we
should sort alignments with identical start positions by score in
reverse order.  Then repeat for the end position.
\nwenddocs{}\nwbegincode{86}\sublabel{NW0-uGjcy-1}\nwmargintag{{\nwtagstyle{}\subpageref{NW0-uGjcy-1}}}\moddef{Remove redundant alignments, Ch.~\ref{ch:sb}~{\nwtagstyle{}\subpageref{NW0-uGjcy-1}}}\endmoddef\nwstartdeflinemarkup\nwusesondefline{\\{NW0-rI1tg-1}}\nwenddeflinemarkup
\LA{}Sort alignments by start, Ch.~\ref{ch:sb}~{\nwtagstyle{}\subpageref{NW0-2MdYsq-1}}\RA{}
\LA{}Delete alignments with identical start, Ch.~\ref{ch:sb}~{\nwtagstyle{}\subpageref{NW0-4E0EaF-1}}\RA{}
\LA{}Sort alignments by end, Ch.~\ref{ch:sb}~{\nwtagstyle{}\subpageref{NW0-3kEXx6-1}}\RA{}
\LA{}Delete alignments with identical end, Ch.~\ref{ch:sb}~{\nwtagstyle{}\subpageref{NW0-47xVRU-1}}\RA{}
\nwused{\\{NW0-rI1tg-1}}\nwendcode{}\nwbegindocs{87}\nwdocspar
We sort the alignments by their start positions using an alignment
slice.
\nwenddocs{}\nwbegincode{88}\sublabel{NW0-2MdYsq-1}\nwmargintag{{\nwtagstyle{}\subpageref{NW0-2MdYsq-1}}}\moddef{Sort alignments by start, Ch.~\ref{ch:sb}~{\nwtagstyle{}\subpageref{NW0-2MdYsq-1}}}\endmoddef\nwstartdeflinemarkup\nwusesondefline{\\{NW0-uGjcy-1}}\nwenddeflinemarkup
sort.Sort(AlSliceStart(alignments))
\nwused{\\{NW0-uGjcy-1}}\nwendcode{}\nwbegindocs{89}\nwdocspar
We import \ty{sort}.
\nwenddocs{}\nwbegincode{90}\sublabel{NW0-2Og5jB-9}\nwmargintag{{\nwtagstyle{}\subpageref{NW0-2Og5jB-9}}}\moddef{Imports, Ch.~\ref{ch:sb}~{\nwtagstyle{}\subpageref{NW0-2Og5jB-1}}}\plusendmoddef\nwstartdeflinemarkup\nwusesondefline{\\{NW0-16JhEl-1}}\nwprevnextdefs{NW0-2Og5jB-8}{\relax}\nwenddeflinemarkup
"sort"
\nwused{\\{NW0-16JhEl-1}}\nwendcode{}\nwbegindocs{91}\nwdocspar
We declare \ty{AlSliceStart}.
\nwenddocs{}\nwbegincode{92}\sublabel{NW0-HzpIk-3}\nwmargintag{{\nwtagstyle{}\subpageref{NW0-HzpIk-3}}}\moddef{Types, Ch.~\ref{ch:sb}~{\nwtagstyle{}\subpageref{NW0-HzpIk-1}}}\plusendmoddef\nwstartdeflinemarkup\nwusesondefline{\\{NW0-16JhEl-1}}\nwprevnextdefs{NW0-HzpIk-2}{NW0-HzpIk-4}\nwenddeflinemarkup
type AlSliceStart []Alignment
\nwused{\\{NW0-16JhEl-1}}\nwendcode{}\nwbegindocs{93}\nwdocspar
We implement the methods \ty{Len}, \ty{Less}, and \ty{Swap} to make
\ty{AlSliceStart} sortable.
\nwenddocs{}\nwbegincode{94}\sublabel{NW0-HmI1L-1}\nwmargintag{{\nwtagstyle{}\subpageref{NW0-HmI1L-1}}}\moddef{Methods, Ch.~\ref{ch:sb}~{\nwtagstyle{}\subpageref{NW0-HmI1L-1}}}\endmoddef\nwstartdeflinemarkup\nwusesondefline{\\{NW0-16JhEl-1}}\nwprevnextdefs{\relax}{NW0-HmI1L-2}\nwenddeflinemarkup
func (a AlSliceStart) Len() int \{
          return len(a)
\}
\LA{}Implement \ty{Less} for \ty{AlSliceStart}, Ch.~\ref{ch:sb}~{\nwtagstyle{}\subpageref{NW0-3eeXUd-1}}\RA{}
func (a AlSliceStart) Swap(i, j int) \{
          a[i], a[j] = a[j], a[i]
\}
\nwalsodefined{\\{NW0-HmI1L-2}\\{NW0-HmI1L-3}}\nwused{\\{NW0-16JhEl-1}}\nwendcode{}\nwbegindocs{95}\nwdocspar
We make sure that for alignments starting at the same position the
highest scoring one comes first.
\nwenddocs{}\nwbegincode{96}\sublabel{NW0-3eeXUd-1}\nwmargintag{{\nwtagstyle{}\subpageref{NW0-3eeXUd-1}}}\moddef{Implement \ty{Less} for \ty{AlSliceStart}, Ch.~\ref{ch:sb}~{\nwtagstyle{}\subpageref{NW0-3eeXUd-1}}}\endmoddef\nwstartdeflinemarkup\nwusesondefline{\\{NW0-HmI1L-1}}\nwenddeflinemarkup
func (a AlSliceStart) Less(i, j int) bool \{
          if a[i].ss == a[j].ss \{
                  return a[i].score > a[j].score
          \} else \{
                  return a[i].ss < a[j].ss
          \}
\}
\nwused{\\{NW0-HmI1L-1}}\nwendcode{}\nwbegindocs{97}\nwdocspar
Wit the alignments sorted by their start and positions and scores, we
can 
\nwenddocs{}\nwbegincode{98}\sublabel{NW0-4E0EaF-1}\nwmargintag{{\nwtagstyle{}\subpageref{NW0-4E0EaF-1}}}\moddef{Delete alignments with identical start, Ch.~\ref{ch:sb}~{\nwtagstyle{}\subpageref{NW0-4E0EaF-1}}}\endmoddef\nwstartdeflinemarkup\nwusesondefline{\\{NW0-uGjcy-1}}\nwenddeflinemarkup
j := 0
if len(alignments) > 0 \{ j = 1 \}
for i := 1; i < len(alignments); i++ \{
          if alignments[i].ss != alignments[i-1].ss \{
                  alignments[j] = alignments[i]
                  j++
          \}
\}
alignments = alignments[:j]
\nwused{\\{NW0-uGjcy-1}}\nwendcode{}\nwbegindocs{99}\nwdocspar
We repeat this procedure for the alignment ends and start again by
sorting the alignments by their end positions using an alignment
slice.
\nwenddocs{}\nwbegincode{100}\sublabel{NW0-3kEXx6-1}\nwmargintag{{\nwtagstyle{}\subpageref{NW0-3kEXx6-1}}}\moddef{Sort alignments by end, Ch.~\ref{ch:sb}~{\nwtagstyle{}\subpageref{NW0-3kEXx6-1}}}\endmoddef\nwstartdeflinemarkup\nwusesondefline{\\{NW0-uGjcy-1}}\nwenddeflinemarkup
sort.Sort(AlSliceEnd(alignments))
\nwused{\\{NW0-uGjcy-1}}\nwendcode{}\nwbegindocs{101}\nwdocspar
We declare \ty{AlSliceEnd}.
\nwenddocs{}\nwbegincode{102}\sublabel{NW0-HzpIk-4}\nwmargintag{{\nwtagstyle{}\subpageref{NW0-HzpIk-4}}}\moddef{Types, Ch.~\ref{ch:sb}~{\nwtagstyle{}\subpageref{NW0-HzpIk-1}}}\plusendmoddef\nwstartdeflinemarkup\nwusesondefline{\\{NW0-16JhEl-1}}\nwprevnextdefs{NW0-HzpIk-3}{NW0-HzpIk-5}\nwenddeflinemarkup
type AlSliceEnd []Alignment
\nwused{\\{NW0-16JhEl-1}}\nwendcode{}\nwbegindocs{103}\nwdocspar
We implement the methods \ty{Len}, \ty{Less}, and \ty{Swap} to make
\ty{AlSliceStart} sortable.
\nwenddocs{}\nwbegincode{104}\sublabel{NW0-HmI1L-2}\nwmargintag{{\nwtagstyle{}\subpageref{NW0-HmI1L-2}}}\moddef{Methods, Ch.~\ref{ch:sb}~{\nwtagstyle{}\subpageref{NW0-HmI1L-1}}}\plusendmoddef\nwstartdeflinemarkup\nwusesondefline{\\{NW0-16JhEl-1}}\nwprevnextdefs{NW0-HmI1L-1}{NW0-HmI1L-3}\nwenddeflinemarkup
func (a AlSliceEnd) Len() int \{
          return len(a)
\}
\LA{}Implement \ty{Less} for \ty{AlSliceEnd}, Ch.~\ref{ch:sb}~{\nwtagstyle{}\subpageref{NW0-wGgQt-1}}\RA{}
func (a AlSliceEnd) Swap(i, j int) \{
          a[i], a[j] = a[j], a[i]
\}
\nwused{\\{NW0-16JhEl-1}}\nwendcode{}\nwbegindocs{105}\nwdocspar
We make sure that for alignments ending at the same position the
highest scoring one comes first.
\nwenddocs{}\nwbegincode{106}\sublabel{NW0-wGgQt-1}\nwmargintag{{\nwtagstyle{}\subpageref{NW0-wGgQt-1}}}\moddef{Implement \ty{Less} for \ty{AlSliceEnd}, Ch.~\ref{ch:sb}~{\nwtagstyle{}\subpageref{NW0-wGgQt-1}}}\endmoddef\nwstartdeflinemarkup\nwusesondefline{\\{NW0-HmI1L-2}}\nwenddeflinemarkup
func (a AlSliceEnd) Less(i, j int) bool \{
          if a[i].se == a[j].se \{
                  return a[i].score > a[j].score
          \} else \{
                  return a[i].se < a[j].se
          \}
\}
\nwused{\\{NW0-HmI1L-2}}\nwendcode{}\nwbegindocs{107}\nwdocspar
Wit the alignments sorted by their end and positions and scores, we
can 
\nwenddocs{}\nwbegincode{108}\sublabel{NW0-47xVRU-1}\nwmargintag{{\nwtagstyle{}\subpageref{NW0-47xVRU-1}}}\moddef{Delete alignments with identical end, Ch.~\ref{ch:sb}~{\nwtagstyle{}\subpageref{NW0-47xVRU-1}}}\endmoddef\nwstartdeflinemarkup\nwusesondefline{\\{NW0-uGjcy-1}}\nwenddeflinemarkup
j = 0
if len(alignments) > 0 \{ j = 1 \}
for i := 1; i < len(alignments); i++ \{
          if alignments[i].se != alignments[i-1].se \{
                  alignments[j] = alignments[i]
                  j++
          \}
\}
alignments = alignments[:j]
\nwused{\\{NW0-uGjcy-1}}\nwendcode{}\nwbegindocs{109}\nwdocspar
We sort the remaining alignments by score.
\nwenddocs{}\nwbegincode{110}\sublabel{NW0-2cS5wy-1}\nwmargintag{{\nwtagstyle{}\subpageref{NW0-2cS5wy-1}}}\moddef{Sort alignments by score, Ch.~\ref{ch:sb}~{\nwtagstyle{}\subpageref{NW0-2cS5wy-1}}}\endmoddef\nwstartdeflinemarkup\nwusesondefline{\\{NW0-1DYjD7-1}}\nwenddeflinemarkup
sort.Sort(AlSliceScore(alignments))
\nwused{\\{NW0-1DYjD7-1}}\nwendcode{}\nwbegindocs{111}\nwdocspar
We declare \ty{AlSliceScore}.
\nwenddocs{}\nwbegincode{112}\sublabel{NW0-HzpIk-5}\nwmargintag{{\nwtagstyle{}\subpageref{NW0-HzpIk-5}}}\moddef{Types, Ch.~\ref{ch:sb}~{\nwtagstyle{}\subpageref{NW0-HzpIk-1}}}\plusendmoddef\nwstartdeflinemarkup\nwusesondefline{\\{NW0-16JhEl-1}}\nwprevnextdefs{NW0-HzpIk-4}{\relax}\nwenddeflinemarkup
type AlSliceScore []Alignment
\nwused{\\{NW0-16JhEl-1}}\nwendcode{}\nwbegindocs{113}\nwdocspar
We implement the \ty{Sort} interface on \ty{AlSliceScore} imposing an
ascending order this time.
\nwenddocs{}\nwbegincode{114}\sublabel{NW0-HmI1L-3}\nwmargintag{{\nwtagstyle{}\subpageref{NW0-HmI1L-3}}}\moddef{Methods, Ch.~\ref{ch:sb}~{\nwtagstyle{}\subpageref{NW0-HmI1L-1}}}\plusendmoddef\nwstartdeflinemarkup\nwusesondefline{\\{NW0-16JhEl-1}}\nwprevnextdefs{NW0-HmI1L-2}{\relax}\nwenddeflinemarkup
func (a AlSliceScore) Len() int \{
          return len(a)
\}
func (a AlSliceScore) Less(i, j int) bool \{
          return a[i].score > a[j].score
\}
func (a AlSliceScore) Swap(i, j int) \{
          a[i], a[j] = a[j], a[i]
\}
\nwused{\\{NW0-16JhEl-1}}\nwendcode{}\nwbegindocs{115}\nwdocspar
The alignments are ready to be printed. Again, we extract the
accessions from the header. Alignments on the reverse strand get their
subject positions switched.
\nwenddocs{}\nwbegincode{116}\sublabel{NW0-4aXeEl-1}\nwmargintag{{\nwtagstyle{}\subpageref{NW0-4aXeEl-1}}}\moddef{Print alignments, Ch.~\ref{ch:sb}~{\nwtagstyle{}\subpageref{NW0-4aXeEl-1}}}\endmoddef\nwstartdeflinemarkup\nwusesondefline{\\{NW0-1R6yr3-1}}\nwenddeflinemarkup
qa := strings.Fields(query.Header())[0]
sa := strings.Fields(subject.Header())[0]
for _, a := range alignments \{
          if !a.forward \{
                  a.ss, a.se = a.se, a.ss
          \}
          fmt.Fprintf(out, "%s\\t%s\\t%d\\t%d\\t%d\\t%d\\t%.1f\\n",
                  qa, sa, a.qs+1, a.qe+1, a.ss+1, a.se+1, a.score)
\}
\nwused{\\{NW0-1R6yr3-1}}\nwendcode{}\nwbegindocs{117}\nwdocspar
We have finished \ty{sblast}, let's test it.

\section*{Testing}
Our testing code has hooks for imports and the testing logic.
\nwenddocs{}\nwbegincode{118}\sublabel{NW0-1OYUWR-1}\nwmargintag{{\nwtagstyle{}\subpageref{NW0-1OYUWR-1}}}\moddef{sblast\_test.go~{\nwtagstyle{}\subpageref{NW0-1OYUWR-1}}}\endmoddef\nwstartdeflinemarkup\nwenddeflinemarkup
package main

import (
          "testing"
          \LA{}Testing imports, Ch.~\ref{ch:sb}~{\nwtagstyle{}\subpageref{NW0-1ZGULc-1}}\RA{}
)

func TestSblast(t *testing.T) \{
          \LA{}Testing, Ch.~\ref{ch:sb}~{\nwtagstyle{}\subpageref{NW0-1wtKzR-1}}\RA{}
\}
\nwnotused{sblast\_test.go}\nwendcode{}\nwbegindocs{119}\nwdocspar
We construct the tests and iterate over them.
\nwenddocs{}\nwbegincode{120}\sublabel{NW0-1wtKzR-1}\nwmargintag{{\nwtagstyle{}\subpageref{NW0-1wtKzR-1}}}\moddef{Testing, Ch.~\ref{ch:sb}~{\nwtagstyle{}\subpageref{NW0-1wtKzR-1}}}\endmoddef\nwstartdeflinemarkup\nwusesondefline{\\{NW0-1OYUWR-1}}\nwenddeflinemarkup
var tests []*exec.Cmd
\LA{}Construct tests, Ch.~\ref{ch:sb}~{\nwtagstyle{}\subpageref{NW0-Ncr9g-1}}\RA{}
for i, test := range tests \{
          \LA{}Run test, Ch.~\ref{ch:sb}~{\nwtagstyle{}\subpageref{NW0-2RU8dU-1}}\RA{}
\}
\nwused{\\{NW0-1OYUWR-1}}\nwendcode{}\nwbegindocs{121}\nwdocspar
We import \ty{exec}.
\nwenddocs{}\nwbegincode{122}\sublabel{NW0-1ZGULc-1}\nwmargintag{{\nwtagstyle{}\subpageref{NW0-1ZGULc-1}}}\moddef{Testing imports, Ch.~\ref{ch:sb}~{\nwtagstyle{}\subpageref{NW0-1ZGULc-1}}}\endmoddef\nwstartdeflinemarkup\nwusesondefline{\\{NW0-1OYUWR-1}}\nwprevnextdefs{\relax}{NW0-1ZGULc-2}\nwenddeflinemarkup
"os/exec"
\nwalsodefined{\\{NW0-1ZGULc-2}}\nwused{\\{NW0-1OYUWR-1}}\nwendcode{}\nwbegindocs{123}\nwdocspar
We test the first seven options listed in Table~\ref{tab:blast}.
\nwenddocs{}\nwbegincode{124}\sublabel{NW0-Ncr9g-1}\nwmargintag{{\nwtagstyle{}\subpageref{NW0-Ncr9g-1}}}\moddef{Construct tests, Ch.~\ref{ch:sb}~{\nwtagstyle{}\subpageref{NW0-Ncr9g-1}}}\endmoddef\nwstartdeflinemarkup\nwusesondefline{\\{NW0-1wtKzR-1}}\nwenddeflinemarkup
\LA{}Test \ty{-a}, Ch.~\ref{ch:sb}~{\nwtagstyle{}\subpageref{NW0-9vAOK-1}}\RA{}
\LA{}Test \ty{-i}, Ch.~\ref{ch:sb}~{\nwtagstyle{}\subpageref{NW0-1qb7sU-1}}\RA{}
\LA{}Test \ty{-w}, Ch.~\ref{ch:sb}~{\nwtagstyle{}\subpageref{NW0-1Avqe-1}}\RA{}
\LA{}Test \ty{-s}, Ch.~\ref{ch:sb}~{\nwtagstyle{}\subpageref{NW0-1Cmn0V-1}}\RA{}
\LA{}Test \ty{-t}, Ch.~\ref{ch:sb}~{\nwtagstyle{}\subpageref{NW0-35M18u-1}}\RA{}
\LA{}Test \ty{-n}, Ch.~\ref{ch:sb}~{\nwtagstyle{}\subpageref{NW0-2audHl-1}}\RA{}
\LA{}Test \ty{-l}, Ch.~\ref{ch:sb}~{\nwtagstyle{}\subpageref{NW0-1Wlaab-1}}\RA{}
\nwused{\\{NW0-1wtKzR-1}}\nwendcode{}\nwbegindocs{125}\nwdocspar
We set the match score from its default of 1 to 2. We use the file
\ty{test.fasta} as query and subject. It contains the \emph{Adh} loci
of \emph{Drosophila melanogaster} and \emph{D. guanche}.
\nwenddocs{}\nwbegincode{126}\sublabel{NW0-9vAOK-1}\nwmargintag{{\nwtagstyle{}\subpageref{NW0-9vAOK-1}}}\moddef{Test \ty{-a}, Ch.~\ref{ch:sb}~{\nwtagstyle{}\subpageref{NW0-9vAOK-1}}}\endmoddef\nwstartdeflinemarkup\nwusesondefline{\\{NW0-Ncr9g-1}}\nwenddeflinemarkup
test := exec.Command("./sblast", "-a", "2",
          "test.fasta", "test.fasta")
tests = append(tests, test)
\nwused{\\{NW0-Ncr9g-1}}\nwendcode{}\nwbegindocs{127}\nwdocspar
We set the mismatch score from default -3 to -2.
\nwenddocs{}\nwbegincode{128}\sublabel{NW0-1qb7sU-1}\nwmargintag{{\nwtagstyle{}\subpageref{NW0-1qb7sU-1}}}\moddef{Test \ty{-i}, Ch.~\ref{ch:sb}~{\nwtagstyle{}\subpageref{NW0-1qb7sU-1}}}\endmoddef\nwstartdeflinemarkup\nwusesondefline{\\{NW0-Ncr9g-1}}\nwenddeflinemarkup
test = exec.Command("./sblast", "-i", "-2",
          "test.fasta", "test.fasta")
tests = append(tests, test)
\nwused{\\{NW0-Ncr9g-1}}\nwendcode{}\nwbegindocs{129}\nwdocspar
We set the word length from default 11 to 20.
\nwenddocs{}\nwbegincode{130}\sublabel{NW0-1Avqe-1}\nwmargintag{{\nwtagstyle{}\subpageref{NW0-1Avqe-1}}}\moddef{Test \ty{-w}, Ch.~\ref{ch:sb}~{\nwtagstyle{}\subpageref{NW0-1Avqe-1}}}\endmoddef\nwstartdeflinemarkup\nwusesondefline{\\{NW0-Ncr9g-1}}\nwenddeflinemarkup
test = exec.Command("./sblast", "-w", "20",
          "test.fasta", "test.fasta")
tests = append(tests, test)
\nwused{\\{NW0-Ncr9g-1}}\nwendcode{}\nwbegindocs{131}\nwdocspar
We reduce the maximum number of idle steps from default 30 to 20.
\nwenddocs{}\nwbegincode{132}\sublabel{NW0-1Cmn0V-1}\nwmargintag{{\nwtagstyle{}\subpageref{NW0-1Cmn0V-1}}}\moddef{Test \ty{-s}, Ch.~\ref{ch:sb}~{\nwtagstyle{}\subpageref{NW0-1Cmn0V-1}}}\endmoddef\nwstartdeflinemarkup\nwusesondefline{\\{NW0-Ncr9g-1}}\nwenddeflinemarkup
test = exec.Command("./sblast", "-s", "20",
          "test.fasta", "test.fasta")
tests = append(tests, test)
\nwused{\\{NW0-Ncr9g-1}}\nwendcode{}\nwbegindocs{133}\nwdocspar
We reduce the threshold score from 50 to 40.
\nwenddocs{}\nwbegincode{134}\sublabel{NW0-35M18u-1}\nwmargintag{{\nwtagstyle{}\subpageref{NW0-35M18u-1}}}\moddef{Test \ty{-t}, Ch.~\ref{ch:sb}~{\nwtagstyle{}\subpageref{NW0-35M18u-1}}}\endmoddef\nwstartdeflinemarkup\nwusesondefline{\\{NW0-Ncr9g-1}}\nwenddeflinemarkup
test = exec.Command("./sblast", "-t", "40",
          "test.fasta", "test.fasta")
tests = append(tests, test)
\nwused{\\{NW0-Ncr9g-1}}\nwendcode{}\nwbegindocs{135}\nwdocspar
We switch from matching with a keyword tree to na\"ive matching.
\nwenddocs{}\nwbegincode{136}\sublabel{NW0-2audHl-1}\nwmargintag{{\nwtagstyle{}\subpageref{NW0-2audHl-1}}}\moddef{Test \ty{-n}, Ch.~\ref{ch:sb}~{\nwtagstyle{}\subpageref{NW0-2audHl-1}}}\endmoddef\nwstartdeflinemarkup\nwusesondefline{\\{NW0-Ncr9g-1}}\nwenddeflinemarkup
test = exec.Command("./sblast", "-n",
          "test.fasta", "test.fasta")
tests = append(tests, test)
\nwused{\\{NW0-Ncr9g-1}}\nwendcode{}\nwbegindocs{137}\nwdocspar
We print the word list.
\nwenddocs{}\nwbegincode{138}\sublabel{NW0-1Wlaab-1}\nwmargintag{{\nwtagstyle{}\subpageref{NW0-1Wlaab-1}}}\moddef{Test \ty{-l}, Ch.~\ref{ch:sb}~{\nwtagstyle{}\subpageref{NW0-1Wlaab-1}}}\endmoddef\nwstartdeflinemarkup\nwusesondefline{\\{NW0-Ncr9g-1}}\nwenddeflinemarkup
test = exec.Command("./sblast", "-l",
          "test.fasta", "test.fasta")
tests = append(tests, test)
\nwused{\\{NW0-Ncr9g-1}}\nwendcode{}\nwbegindocs{139}\nwdocspar
When running \ty{sblast}, we compare what we get with what we want,
which is contained in results files \ty{r1.txt}, \ty{r2.txt}, and so
on.
\nwenddocs{}\nwbegincode{140}\sublabel{NW0-2RU8dU-1}\nwmargintag{{\nwtagstyle{}\subpageref{NW0-2RU8dU-1}}}\moddef{Run test, Ch.~\ref{ch:sb}~{\nwtagstyle{}\subpageref{NW0-2RU8dU-1}}}\endmoddef\nwstartdeflinemarkup\nwusesondefline{\\{NW0-1wtKzR-1}}\nwenddeflinemarkup
get, err := test.Output()
if err != nil \{ t.Errorf("couldn't run %s\\n", test) \}
f := "r" + strconv.Itoa(i+1) + ".txt"
want, err := ioutil.ReadFile(f)
if err != nil \{ t.Errorf("couldn't open %s\\n", f) \}
if !bytes.Equal(get, want) \{
          t.Errorf("get:\\n%s\\nwant:\\n%s\\n",
                  string(get), string(want))
\}
\nwused{\\{NW0-1wtKzR-1}}\nwendcode{}\nwbegindocs{141}\nwdocspar
We import \ty{strconv}, \ty{ioutil}, and \ty{bytes}.
\nwenddocs{}\nwbegincode{142}\sublabel{NW0-1ZGULc-2}\nwmargintag{{\nwtagstyle{}\subpageref{NW0-1ZGULc-2}}}\moddef{Testing imports, Ch.~\ref{ch:sb}~{\nwtagstyle{}\subpageref{NW0-1ZGULc-1}}}\plusendmoddef\nwstartdeflinemarkup\nwusesondefline{\\{NW0-1OYUWR-1}}\nwprevnextdefs{NW0-1ZGULc-1}{\relax}\nwenddeflinemarkup
"strconv"
"io/ioutil"
"bytes"
\nwused{\\{NW0-1OYUWR-1}}\nwendcode{}

\nwixlogsorted{c}{{Adjust alignment end? Ch.~\ref{ch:sb}}{NW0-3W96lC-1}{\nwixu{NW0-3FnixI-1}\nwixd{NW0-3W96lC-1}}}%
\nwixlogsorted{c}{{Adjust alignment start? Ch.~\ref{ch:sb}}{NW0-vHJSe-1}{\nwixu{NW0-35cUih-1}\nwixd{NW0-vHJSe-1}}}%
\nwixlogsorted{c}{{Align query, Ch.~\ref{ch:sb}}{NW0-1R6yr3-1}{\nwixu{NW0-3uTedU-1}\nwixd{NW0-1R6yr3-1}}}%
\nwixlogsorted{c}{{Analyze query, Ch.~\ref{ch:sb}}{NW0-3uTedU-1}{\nwixu{NW0-2otCSW-1}\nwixd{NW0-3uTedU-1}}}%
\nwixlogsorted{c}{{Calculate alignments, Ch.~\ref{ch:sb}}{NW0-1DYjD7-1}{\nwixu{NW0-3nuSaQ-3}\nwixd{NW0-1DYjD7-1}}}%
\nwixlogsorted{c}{{Collect option values, Ch.~\ref{ch:sb}}{NW0-45Dubr-1}{\nwixu{NW0-1n3jf6-1}\nwixd{NW0-45Dubr-1}}}%
\nwixlogsorted{c}{{Compare current pair of residues, Ch.~\ref{ch:sb}}{NW0-4WxDgc-1}{\nwixu{NW0-35cUih-1}\nwixd{NW0-4WxDgc-1}\nwixu{NW0-3FnixI-1}}}%
\nwixlogsorted{c}{{Construct keyword tree, Ch.~\ref{ch:sb}}{NW0-4Wz7Gt-1}{\nwixu{NW0-2zPa6i-1}\nwixd{NW0-4Wz7Gt-1}}}%
\nwixlogsorted{c}{{Construct patterns, Ch.~\ref{ch:sb}}{NW0-UZ3nD-1}{\nwixu{NW0-2zPa6i-1}\nwixd{NW0-UZ3nD-1}}}%
\nwixlogsorted{c}{{Construct tests, Ch.~\ref{ch:sb}}{NW0-Ncr9g-1}{\nwixu{NW0-1wtKzR-1}\nwixd{NW0-Ncr9g-1}}}%
\nwixlogsorted{c}{{Convert matches to alignments, Ch.~\ref{ch:sb}}{NW0-2JbbBG-1}{\nwixu{NW0-2zPa6i-1}\nwixd{NW0-2JbbBG-1}}}%
\nwixlogsorted{c}{{Declare options, Ch.~\ref{ch:sb}}{NW0-225M11-1}{\nwixu{NW0-4WmA2g-1}\nwixd{NW0-225M11-1}}}%
\nwixlogsorted{c}{{Delete alignments with identical end, Ch.~\ref{ch:sb}}{NW0-47xVRU-1}{\nwixu{NW0-uGjcy-1}\nwixd{NW0-47xVRU-1}}}%
\nwixlogsorted{c}{{Delete alignments with identical start, Ch.~\ref{ch:sb}}{NW0-4E0EaF-1}{\nwixu{NW0-uGjcy-1}\nwixd{NW0-4E0EaF-1}}}%
\nwixlogsorted{c}{{Exact match with keyword tree, Ch.~\ref{ch:sb}}{NW0-2zPa6i-1}{\nwixu{NW0-1uN9bQ-1}\nwixd{NW0-2zPa6i-1}}}%
\nwixlogsorted{c}{{Exact matching, Ch.~\ref{ch:sb}}{NW0-1uN9bQ-1}{\nwixu{NW0-1DYjD7-1}\nwixd{NW0-1uN9bQ-1}}}%
\nwixlogsorted{c}{{Extend alignments, Ch.~\ref{ch:sb}}{NW0-2p7uJM-1}{\nwixu{NW0-1DYjD7-1}\nwixd{NW0-2p7uJM-1}}}%
\nwixlogsorted{c}{{Filter alignments, Ch.~\ref{ch:sb}}{NW0-rI1tg-1}{\nwixu{NW0-1DYjD7-1}\nwixd{NW0-rI1tg-1}}}%
\nwixlogsorted{c}{{Functions, Ch.~\ref{ch:sb}}{NW0-3nuSaQ-1}{\nwixu{NW0-16JhEl-1}\nwixd{NW0-3nuSaQ-1}\nwixd{NW0-3nuSaQ-2}\nwixd{NW0-3nuSaQ-3}}}%
\nwixlogsorted{c}{{Implement \ty{Less} for \ty{AlSliceEnd}, Ch.~\ref{ch:sb}}{NW0-wGgQt-1}{\nwixu{NW0-HmI1L-2}\nwixd{NW0-wGgQt-1}}}%
\nwixlogsorted{c}{{Implement \ty{Less} for \ty{AlSliceStart}, Ch.~\ref{ch:sb}}{NW0-3eeXUd-1}{\nwixu{NW0-HmI1L-1}\nwixd{NW0-3eeXUd-1}}}%
\nwixlogsorted{c}{{Imports, Ch.~\ref{ch:sb}}{NW0-2Og5jB-1}{\nwixu{NW0-16JhEl-1}\nwixd{NW0-2Og5jB-1}\nwixd{NW0-2Og5jB-2}\nwixd{NW0-2Og5jB-3}\nwixd{NW0-2Og5jB-4}\nwixd{NW0-2Og5jB-5}\nwixd{NW0-2Og5jB-6}\nwixd{NW0-2Og5jB-7}\nwixd{NW0-2Og5jB-8}\nwixd{NW0-2Og5jB-9}}}%
\nwixlogsorted{c}{{Iterate across queries, Ch.~\ref{ch:sb}}{NW0-2otCSW-1}{\nwixu{NW0-3nuSaQ-1}\nwixd{NW0-2otCSW-1}}}%
\nwixlogsorted{c}{{Look for pattern, Ch.~\ref{ch:sb}}{NW0-1aqFvq-1}{\nwixu{NW0-4CQMjJ-1}\nwixd{NW0-1aqFvq-1}}}%
\nwixlogsorted{c}{{Main function, Ch.~\ref{ch:sb}}{NW0-4WmA2g-1}{\nwixu{NW0-16JhEl-1}\nwixd{NW0-4WmA2g-1}}}%
\nwixlogsorted{c}{{Methods, Ch.~\ref{ch:sb}}{NW0-HmI1L-1}{\nwixu{NW0-16JhEl-1}\nwixd{NW0-HmI1L-1}\nwixd{NW0-HmI1L-2}\nwixd{NW0-HmI1L-3}}}%
\nwixlogsorted{c}{{Na\"ive exact matching, Ch.~\ref{ch:sb}}{NW0-4CQMjJ-1}{\nwixu{NW0-1uN9bQ-1}\nwixd{NW0-4CQMjJ-1}}}%
\nwixlogsorted{c}{{Parse input files, Ch.~\ref{ch:sb}}{NW0-3CnRTg-1}{\nwixu{NW0-4WmA2g-1}\nwixd{NW0-3CnRTg-1}}}%
\nwixlogsorted{c}{{Parse options, Ch.~\ref{ch:sb}}{NW0-1n3jf6-1}{\nwixu{NW0-4WmA2g-1}\nwixd{NW0-1n3jf6-1}}}%
\nwixlogsorted{c}{{Print alignments, Ch.~\ref{ch:sb}}{NW0-4aXeEl-1}{\nwixu{NW0-1R6yr3-1}\nwixd{NW0-4aXeEl-1}}}%
\nwixlogsorted{c}{{Print word list, Ch.~\ref{ch:sb}}{NW0-4ax9cN-1}{\nwixu{NW0-3uTedU-1}\nwixd{NW0-4ax9cN-1}}}%
\nwixlogsorted{c}{{Remove low-scoring alignments, Ch.~\ref{ch:sb}}{NW0-1lD71o-1}{\nwixu{NW0-rI1tg-1}\nwixd{NW0-1lD71o-1}}}%
\nwixlogsorted{c}{{Remove redundant alignments, Ch.~\ref{ch:sb}}{NW0-uGjcy-1}{\nwixu{NW0-rI1tg-1}\nwixd{NW0-uGjcy-1}}}%
\nwixlogsorted{c}{{Retrieve arguments, Ch.~\ref{ch:sb}}{NW0-3worYP-1}{\nwixu{NW0-3nuSaQ-1}\nwixd{NW0-3worYP-1}}}%
\nwixlogsorted{c}{{Run test, Ch.~\ref{ch:sb}}{NW0-2RU8dU-1}{\nwixu{NW0-1wtKzR-1}\nwixd{NW0-2RU8dU-1}}}%
\nwixlogsorted{c}{{sblast.go}{NW0-16JhEl-1}{\nwixd{NW0-16JhEl-1}}}%
\nwixlogsorted{c}{{sblast\_test.go}{NW0-1OYUWR-1}{\nwixd{NW0-1OYUWR-1}}}%
\nwixlogsorted{c}{{Search with keyword tree, Ch.~\ref{ch:sb}}{NW0-1YDNRQ-1}{\nwixu{NW0-2zPa6i-1}\nwixd{NW0-1YDNRQ-1}}}%
\nwixlogsorted{c}{{Set usage, Ch.~\ref{ch:sb}}{NW0-ys5c2-1}{\nwixu{NW0-4WmA2g-1}\nwixd{NW0-ys5c2-1}}}%
\nwixlogsorted{c}{{Sort alignments by end, Ch.~\ref{ch:sb}}{NW0-3kEXx6-1}{\nwixu{NW0-uGjcy-1}\nwixd{NW0-3kEXx6-1}}}%
\nwixlogsorted{c}{{Sort alignments by score, Ch.~\ref{ch:sb}}{NW0-2cS5wy-1}{\nwixu{NW0-1DYjD7-1}\nwixd{NW0-2cS5wy-1}}}%
\nwixlogsorted{c}{{Sort alignments by start, Ch.~\ref{ch:sb}}{NW0-2MdYsq-1}{\nwixu{NW0-uGjcy-1}\nwixd{NW0-2MdYsq-1}}}%
\nwixlogsorted{c}{{Test \ty{-a}, Ch.~\ref{ch:sb}}{NW0-9vAOK-1}{\nwixu{NW0-Ncr9g-1}\nwixd{NW0-9vAOK-1}}}%
\nwixlogsorted{c}{{Test \ty{-i}, Ch.~\ref{ch:sb}}{NW0-1qb7sU-1}{\nwixu{NW0-Ncr9g-1}\nwixd{NW0-1qb7sU-1}}}%
\nwixlogsorted{c}{{Test \ty{-l}, Ch.~\ref{ch:sb}}{NW0-1Wlaab-1}{\nwixu{NW0-Ncr9g-1}\nwixd{NW0-1Wlaab-1}}}%
\nwixlogsorted{c}{{Test \ty{-n}, Ch.~\ref{ch:sb}}{NW0-2audHl-1}{\nwixu{NW0-Ncr9g-1}\nwixd{NW0-2audHl-1}}}%
\nwixlogsorted{c}{{Test \ty{-s}, Ch.~\ref{ch:sb}}{NW0-1Cmn0V-1}{\nwixu{NW0-Ncr9g-1}\nwixd{NW0-1Cmn0V-1}}}%
\nwixlogsorted{c}{{Test \ty{-t}, Ch.~\ref{ch:sb}}{NW0-35M18u-1}{\nwixu{NW0-Ncr9g-1}\nwixd{NW0-35M18u-1}}}%
\nwixlogsorted{c}{{Test \ty{-w}, Ch.~\ref{ch:sb}}{NW0-1Avqe-1}{\nwixu{NW0-Ncr9g-1}\nwixd{NW0-1Avqe-1}}}%
\nwixlogsorted{c}{{Testing imports, Ch.~\ref{ch:sb}}{NW0-1ZGULc-1}{\nwixu{NW0-1OYUWR-1}\nwixd{NW0-1ZGULc-1}\nwixd{NW0-1ZGULc-2}}}%
\nwixlogsorted{c}{{Testing, Ch.~\ref{ch:sb}}{NW0-1wtKzR-1}{\nwixu{NW0-1OYUWR-1}\nwixd{NW0-1wtKzR-1}}}%
\nwixlogsorted{c}{{Types, Ch.~\ref{ch:sb}}{NW0-HzpIk-1}{\nwixu{NW0-16JhEl-1}\nwixd{NW0-HzpIk-1}\nwixd{NW0-HzpIk-2}\nwixd{NW0-HzpIk-3}\nwixd{NW0-HzpIk-4}\nwixd{NW0-HzpIk-5}}}%
\nwixlogsorted{c}{{Walk left, Ch.~\ref{ch:sb}}{NW0-35cUih-1}{\nwixu{NW0-2p7uJM-1}\nwixd{NW0-35cUih-1}}}%
\nwixlogsorted{c}{{Walk right, Ch.~\ref{ch:sb}}{NW0-3FnixI-1}{\nwixu{NW0-2p7uJM-1}\nwixd{NW0-3FnixI-1}}}%

